A kernel specifies functionality to be executed on a device.
It is specified using the declaration specifier \textit{\_\_global\_\_}, so that it can be distinguished from a regular C function.
When a kernel is launched, it is executed \textit{N} times in parallel by \textit{N} threads on the device.
The kernel is launched from the host, where it is specified how many \cuda{} threads to execute the kernel.
The launch of a kernel is specified by "$kernel\_name<<< number\_of\_blocks, number\_of\_threads\_in\_each\_block >>>$"
An example of a kernel and how it is being launched can be seen in \autoref{lst:kernel}.
\begin{lstlisting}[language=C,caption={Kernel example},label=lst:kernel]
__global__ void hello_world(float * d_out, float * d_in){
	print('hello world')
}
int main(int argc, char ** argv) {
	hello_world<<<1, 100>>>(d_out, d_in);
}
\end{lstlisting}
The kernel is launched with a single thread block with 100 threads.