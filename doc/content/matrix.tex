Matrix multiplication has numerous applications and are often used in matrix algebra and in geometry based computing.
Many matrices in such applications have a lot of entries with the value zero.
It is said that a matrix is \textit{dense} if all elements in the matrix are stored including zero valued inputs.
Conversely, a matrix is \textit{sparse} if the zero valued elements in the matrix are omitted.

A way to represent a sparse matrix is done by a \textit{compressed sparse row}.
Say that a dense matrix of 9 elements is represented as follows
\begin{gather}
	\begin{bmatrix}
		a & 0 & b \\[0.3em]
		c & d & e \\[0.3em]
		0 & 0 & f
	\end{bmatrix}
	\label{mat-dense}
\end{gather}
Such a matrix can be represented as a sparse matrix in \textit{compressed sparse row} format, abbreviated as \textbf{CSR} format.
The CSR format requires three vectors that together represents the sparse matrix.
The three vectors are in this notation named as $V_v$, $C_v$ and $R_v$.
The first vector is called the \textit{value} vector.
This vector simply represent all non-zero data.
In the case of matrix \ref{mat-dense} the value vector, $V_v$, is
\begin{gather}
	V_v =
	\begin{bmatrix}
		a & b & c & d & e & f\\[0.3em]
	\end{bmatrix}
\end{gather}
The second vector, $C_v$, records which column each of the element from $V_v$ came from.
In the example from matrix \ref{mat-dense} $C_v$ becomes
\begin{gather}
	C_v =
	\begin{bmatrix}
		0 & 2 & 0 & 1 & 2 & 2\\[0.3em]
	\end{bmatrix}
	\label{mat-vvector}
\end{gather}
since element \textit{a} is in column one, element \textit{b} is in column two and so on.
The final vector, $R_v$, indicates which element start each row.
This vector is also called the \textit{row-pointer}.
In the example from matrix \ref{mat-vvector} the row-pointer is
\begin{gather}
	R_v =
	\begin{bmatrix}
		0 & 2 & 5\\[0.3em]
	\end{bmatrix}
\end{gather}
since element \textit{a} starts in the first row and the index in $V_v$ is "0" and so on.
It is now possible to reconstruct this sparse matrix represented by the three vectors $V_v$, $C_v$ and $R_v$.

Doing matrix multiplication on a dense matrix results in a lot of multiplications that have no effect since many of the elements are zero.
This is not computationally efficient so the idea is to use a sparse matrix representation for doing the computation.
This is achieved by doing \textit{sparse matrix/dense vector multiplication}, abbreviated as \textbf{SpMv}.
Formally SpMv is in the form $\mathbf{y=Ax}$, where the input matrix $\mathbf{A}$ is sparse and the input vector $\mathbf{x}$ and the output vector $\mathbf{y}$ are dense.
As an example take matrix \ref{mat-dense} multiplied with another matrix
\begin{gather}
	\begin{bmatrix}
		a & 0 & b \\[0.3em]
		c & d & e \\[0.3em]
		0 & 0 & f
	\end{bmatrix}
	\begin{bmatrix}
		x \\[0.3em]
		y \\[0.3em]
		z 
	\end{bmatrix}
	=
	\begin{bmatrix}
	ax + 0y + bz \\[0.3em]
	cx + dy + ez \\[0.3em]
	0x + 0y + fz
	\end{bmatrix}
	\label{mat-ex}
\end{gather}
There are 4 steps to compute this using the CSR format for a sparse matrix representation:
\begin{center}
	\fbox{
		\begin{tabular}{p{40pt} p{220pt}}
			\textbf{Step 1} & Create a segmented representation of the value vector by using $V_v$ and $R_v$ \\
			\textbf{Step 2} & Gather vector values using columns \\
			\textbf{Step 3} & Pairwise multiply the resulting vectors from step 1 and 2 \\
			\textbf{Step 3} & Do an exclusive segmented sum scan on the resulting vector from step 3
		\end{tabular}
	}
	\captionof{table}{Steps to SpMv}
	\label{alg-spmv}
\end{center}
An example of doing a actual matrix multiplication as seen in \autoref{mat-ex} using a CSR format and the steps described in \autoref{alg-spmv} is seen in \autoref{alg-mulex}.
\begin{center}
	\fbox{
		\begin{tabular}{p{40pt} p{220pt}}
			\textbf{Step 1} & $\begin{bmatrix}
								a & b & | & c & d & e & | & f \\[0.3em]
							  \end{bmatrix}$ \\
			\textbf{Step 2} & $\begin{bmatrix}
								x & z & x & y & z & y \\[0.3em]
						      \end{bmatrix}$ \\
			\textbf{Step 3} & $\begin{bmatrix}
								a\cdot x\ \ b\cdot z & | & c\cdot x\ \ d\cdot y\ \ e\cdot z & | & f\cdot y  \\[0.3em]
							  \end{bmatrix}$ \\
			\textbf{Step 4} & $\begin{bmatrix}
								a\cdot x+b\cdot z & | & c\cdot x+d\cdot y+e\cdot z & | & f\cdot y  \\[0.3em]
							  \end{bmatrix}$ \\
		\end{tabular}
	}
	\captionof{table}{Example of SpMv multiplication using CSR format}
	\label{alg-mulex}
\end{center}
The first step makes a segmented representation of the value vector where each segment represents one row of the sparse matrix.
Note that each segment is found by using $R_v$ and is represented by a vertical line ("$|$").
In the second step each input in the $C_v$ indicates which element it need to be multiplied with.
In step three, pairwise multiplications are done.
For this the \textit{map} pattern is applied as described in \autoref{sec-map}.
In the final step each partial product within a segment need to be added up by doing a exclusive segmented sum scan as described in \autoref{sec:al_scan}.
Using SpMv exploits a lot a data parallelism and is also efficient since its computations are based on sparse matrices.

Computing SpMv in parallel can be done by assigning a row or an element to each thread.
In the case where each row is assigned to a thread, also called thread per row, the thread makes a dot product of a row in the sparse input matrix and the dense column vector.
In the case where each thread is assigned a single element, also called thread per element, a partial product is created.
In this case each thread needs to communicate with other threads to compute the correct SpMv.

%TODO: maybe make an example of row pr thread or element pr thread. notice that there is are code examples in the video! Maybe make a performance ananlysis if there is time?
