Matrix multiplication has numerous applications and are often used in matrix algebra and in geometry based computing.
Many matrices in such applications have a lot of entries with the value zero.
It is said that a matrix is \textit{dense} if all elements in the matrix are stored including zero valued inputs.
Conversely, a matrix is \textit{sparse} if the zero valued elements in the matrix are omitted.

A way to represent a sparse matrix is done by a \textit{compressed sparse row}.
Say that a dense matrix of 9 elements is represented as follows
\begin{gather}
	\begin{bmatrix}
		a & 0 & b \\[0.3em]
		c & d & e \\[0.3em]
		0 & 0 & f
	\end{bmatrix}
	\label{mat-dense}
\end{gather}
Such a matrix can be represented as a sparse matrix in \textit{compressed sparse row} format, abbreviated as \textbf{CSV} format.
The CSR format requires three vectors that together represents the sparse matrix.
The three vectors are in this notation named as $V_v$, $C_v$ and $R_v$.
The first vector is called the \textit{value} vector.
This vector simply represent all non-zero data.
In the case of matrix \ref{mat-dense} the value vector, $V_v$, is
\begin{gather}
	V_v =
	\begin{bmatrix}
		a & b & c & d & e & f\\[0.3em]
	\end{bmatrix}
\end{gather}
The second vector, $C_v$, records which column each of the element from $V_v$ came from.
In the example from matrix \ref{mat-dense} $C_v$ becomes
\begin{gather}
	C_v =
	\begin{bmatrix}
		0 & 2 & 0 & 1 & 2 & 2\\[0.3em]
	\end{bmatrix}
	\label{mat-vvector}
\end{gather}
since element \textit{a} is in column one, element \textit{b} is in column two and so on.
The final vector, $R_v$, indicates which element start each row.
This vector is also called the \textit{row-pointer}.
In the example from matrix \ref{mat-vvector} the row-pointer is
\begin{gather}
	R_v =
	\begin{bmatrix}
		0 & 2 & 5\\[0.3em]
	\end{bmatrix}
\end{gather}
since element \textit{a} starts in the first row and the index in $V_v$ is "0" and so on.
It is now possible to reconstruct this sparse matrix represented by the three vectors $V_v$, $C_v$ and $R_v$.

Doing matrix multiplication on a dense matrix results in a lot of multiplications that have no effect since many of the elements are zero.
This is not computationally efficient so the idea is to use a sparse matrix representation for doing the computation.
This is achieved by doing \textit{sparse matrix/dense vector multiplication}, abbreviated as \textbf{SpMv}.
Formally SpMv is in the form $\mathbf{y=Ax}$, where the input matrix $\mathbf{A}$ is sparse and the input vector $\mathbf{x}$ and the output vector $\mathbf{y}$ are dense.
As an example take matrix \ref{mat-dense} multiplied with another matrix
\begin{gather}
	\begin{bmatrix}
		a & 0 & b \\[0.3em]
		c & d & e \\[0.3em]
		0 & 0 & f
	\end{bmatrix}
	\begin{bmatrix}
		x \\[0.3em]
		y \\[0.3em]
		z 
	\end{bmatrix}
	=
	\begin{bmatrix}
	ax + 0y + bz \\[0.3em]
	cx + dy + ez \\[0.3em]
	0x + 0y + fz
	\end{bmatrix}
	\label{mat-ex}
\end{gather}
There are 4 steps to compute this using the CSV format for a sparse matrix representation:
\begin{center}
	\fbox{
		\begin{tabular}{p{40pt} p{220pt}}
			\textbf{Step 1} & Use predicate on input elements, $f_{P}(S)$, and translate the results into zeros and ones \\
			\textbf{Step 2} & Do an exclusive sum scan on the zeros and ones array, $f_{P}(S_{rand})_{new}$, from step 1. The output is scatter addresses for the compacted array.\\
			\textbf{Step 3} & Scatter the input array into the output array using the scatter addresses.
		\end{tabular}
	}
	\captionof{table}{Steps to compact}
	\label{alg-spmv}
\end{center}

Lesson 4 and 6.1