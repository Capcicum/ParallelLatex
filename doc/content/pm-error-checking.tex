In \cuda{} all synchronous operations returns an error code.
This error code can be checked with \textit{cudaPeekAtLastError()} or \textit{cudaGetLastError()}.
It is a different story for asynchronous functions, as these cannot return the error code immediately, because they have not finished its execution when the function returns.
The only way to retrieve the error code is to call \textit{cudaDeviceSynchronize()} right after the asynchronous function call.
This functions waits for the asynchronous call to finish and afterwards is it possible to retrieve the error code in the same way as for synchronous function calls.
The use of error checking can be beneficial during the development of \cuda{} programs to explicitly check if any errors occur \cite{cuda:programmingguide}. 
 