The motivation behind the first GPUs, emerging at the end of the 1990s, was a high demand of GPU accelerated 3D graphics.
The early versions of GPU hardware architectures was narrowly specialized to perform a fixed series of standard operations onto the data it was provided.
This resulted in large performance improvements in comparison to the general purpose CPU.
However the drawback was that 3D rendering were performed using fixed-function pipelines (FFPs).
Thus, configuration of which function to perform was not a possibility, only the parameters could be changed.
The functions performed by FFPs handling operations which are \textit{applying lightning}, \textit{coloring} and \textit{adding textures to shapes} are defined as shaders.
For early FFPs, two shaders existed, a shader responsible for vertexes(triangles) and a shader for pixels.

% Consider: (Rewrite)
%The first stage in the fixed-function graphics pipeline converted triangle data originating %from the CPU to a form that the graphics hardware could understand. The next pipeline stages %colorized and applied textures to the triangles, rasterized the 3D-scene to a 2D-image and %colorized the pixels in the triangles. In the final stage in the pipeline, the output image %was stored in a frame buffer where it resided until it was displayed on the screen

The first step towards the GPGPU was made to increase the configuration possibilities.
This was carried out by changing the GPU architecture to provide a dedicated pool of processing cores for each of the two shaders.
Allowing for customization of the pipeline so that custom algorithms can be performed on each element.

However, often only one of the two shaders were performed at a time, resulting in that half %TODO Cuda/DirectX10 -> Only one type of core -> GPGPU

