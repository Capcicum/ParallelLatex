The \cuda{} programming model, is a heterogeneous model where both the CPU and the GPU are used.
In this context \textit{host} refers to the CPU and \textit{device} refers to the GPU and their respective memory.
The code run on the host also manages memory on the device and launches \textit{kernels} to be executed on the device.
The host and device are physically separated, which means that the host execution can run in parallel with the kernel execution on the device.
Kernels can briefly be described as functionality being executed by multiple threads in parallel on the device.

\noindent A typical sequence of operations performed in the \cuda{} programming model is as follows:
\begin{enumerateSmall}
	\item Allocate and initialize host and device memory
	\item Transfer data from host to device
	\item Execute kernel(s) on the device
	\item Transfer result(s) from device to host
\end{enumerateSmall}
The operations involved is described in details in the following sections.
The \cuda{} examples are written in "\cuda{} C", which is an extension to ANSI C, but the principles are generic and can be applied in all \cuda{} programming languages.