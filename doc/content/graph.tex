Graph theory is used many applications and it is possible to implement graph traversals in parallel.
This section does not describe graph theory in depth, but explains some basic notions of graph theory and then describes how graph traversals can be implemented in parallel.

There are two types of graphs: undirected and directed graphs.
This section only considers undirected graphs.
An undirected graph \textit{G} is represented as a pair \textit{(V, E)}, where 
\begin{itemizeSmall}
	\item[\textbf{V}], $V\notin \{\}$ and is called the set of vertices. A vertex can also be denoted as a "node".
	\item[\textbf{E}], set of unordered pairs of vertices and is called the set of edges.
	\item[\textbf{(u, v)}] is an edge which has two endpoints \textit{u} and \textit{v} where both are vertices.
\end{itemizeSmall}
An example of a undirected graph is depicted in bla bla FIG.

A graph can be traversed and this means that each node of the graph is visited.
There are two types of graph traversals: breadth-first search, \textbf{BFS}, and depth-first search, \textbf{DFS}.
\begin{description}
	\item[Breadth-first search] A breadth-first search traversal begins at a particular node and visits all neighboring nodes that are one hop away and labels them as visited. Then all nodes which have been visited one hop away from the starting point are now visiting all their one-hop-away-neighbors. This is done iteratively until all nodes have been labeled and thus visited. An example is seen in bla. bla. minipage FIG
	\item[Depth-first search] A depth-first search traversal begins at a particular node and picks a neighboring node which has not been visited yet. This node is then labeled as visited and yet another depth-first traversal is done from this node. If a node does not have an unvisited neighbor then it goes to a previously labeled node and picks an unlabeled neighboring node. This process is done iteratively until all nodes have been labeled as visited. The process of a depth first traversal is shown in bla bla. FIG (maybe do a minipage here showing the example)
\end{description}
In general the \textit{DFS} is more memory efficient since it requires less state, but \textit{BFS} exposes more parallelism during the iterative traversal stages.
For this reason only \textit{BFS} is considered in parallel.

Explain the concept of a frontier in a graph.
Calculate distance from root (starting point) (min and max as functions of n)
Parallel algorithms to solve BFS.
Work complexity.
Code.
Optimizing runtime for graph traversal (lesson 6.2) very detailed and complex to describe in writing. Beware

Lesson 6.1