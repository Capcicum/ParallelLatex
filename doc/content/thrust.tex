Until now this report has described many parallel patterns, algorithms and applications.
Common for all are that they can be very cumbersome to implement and are prone to many  errors which are difficult to find such as wrong memory allocation, off-by-one errors, complex and unreadable source code, high algorithm complexity and deep understanding of CUDA and the underlying architecture.
\\\\
Thrust is a C++ library which enables programming of high performance applications with minimal programming efforts.
It is based on the Standard Template Library (STL) and gives a rich collection of data parallel primitives such as scan, sort and reduce.
Since the desired computation is described in a high abstraction level, Thrust will try to optimize and select the most efficient implementation.

Thrust is basically built upon three building blocks which are vectors, algorithms and iterators.
There are two types of vectors in Thrust: \textit{host\_vector} and \textit{device\_vector}.
Host vectors are stored in host memory while device vectors are stored in device memory.
This is makes allocation and copying between host and device relatively straight forward.
An example of this is seen in \autoref{lst:thrust-vec}.
\begin{lstlisting}[language=C,caption={Vectors in Thrust},label=lst:thrust-vec]
#include <thrust/host_vector.h> 
#include <thrust/device_vector.h> 
int main(void) { 
	thrust::host_vector<int> H(4); 
	...// Initialize H
	thrust::device_vector<int> D = H; 
	...// Modify D directly
	return 0;
}
\end{lstlisting}
There are many benefits of using Thrust vectors.
For instance the vectors can be resized and copying between host and device is straightforward by the overloaded "="-operator.
Furthermore, both host and device vectors are deleted and the memory is freed once the function returns.
Thrust also gives useful functions for copying and initializing data on the host and device.
Examples of such functions are \textit{thrust::fill}, \textit{thrust::sequence} and \textit{thrust::copy}.
It is still possible to extract raw CUDA device pointers from Thrust vectors

Thrust enables easy usage of parallel communication patterns and algorithms as described in \autoref{ch-patterns} and \autoref{ch:algorithms}.
Following are some examples of such patterns and algorithms which are supported in Thrust:
\begin{itemizeSmall}
	\item thrust::reduce
	\item thrust::inclusive\_scan
	\item thrust::exclusive\_scan
	\item thrust::sort
	\item thrust::sort\_by\_key
	\item thrust::stable\_sort
\end{itemizeSmall}
For instance an exclusive scan is seen in \autoref{lst:thrust-scan}

\begin{lstlisting}[language=C,caption={Exclusive scan in Thrust},label=lst:thrust-scan]
#include <thrust/scan.h> 
...
int data[6] = {1, 0, 2, 2, 1, 3}; 
thrust::exclusive_scan(data, data + 6, data); // data becomes {0, 1, 1, 3, 5, 6}
...
\end{lstlisting}
This simplifies a lot TODO REF TO CODE EXAMPLE HERE IF TIME.

Using the Thrust library also enables the usage of fairly complex iterators.
Some of the iterators in Thrust are:
\begin{description}
	\item[Constant iterator] The \textit{constant\_iterator} returns the same value whenever it is accessed.
	\item[Counting iterator] The \textit{counting\_iterator} increments its value by one with respect to the initialization value.
	\item[Permutation iterator] The \textit{make\_permutation\_iterator} allows to fuse for example gather and scatter operations into a single iterator. This is a rather complex subject so it will not be discussed in depth in this report.
	\item[Zip iterator] The \textit{zip\_iterator} takes multiple input sequences and returns a sequence of tuples. It is rather useful if for instance one would like to iterate over two distinct data types in a single loop.
\end{description}
All operators in Thrust are accessed like arrays.
\\\\
This was only an introduction to Thrust and which features there are that can ease CUDA application development, where many of the core CUDA details are abstracted away.
This of course makes it easier and faster to prototype GPU applications, but at the cost of total control of the GPU.
For more info of Thrust see \url{http://docs.nvidia.com/cuda/thrust/index.html}. (should this be a ref in stead?)