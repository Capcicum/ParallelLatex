One of the simplest parallel sorting algorithms is odd-even sort also know as odd-even transposition sort and brick sort. It is a comparison sort and can be seen as the parallel version of the simple serial bubble sort. The odd-even sort consists of two phases, an odd, and an even. In each phase, each odd or even index array pair is compared and swapped if necessary. An visual representation of the odd-even sort is seen in \cref{fig:sort_odd_even}.

\begin{figure}[ht]
	\centering
	\fbox{
		\includegraphics[width=0.5\textwidth]{figs/algorithm/sort_odd_even.png}}
	\caption{Odd-even sort, orange blocks are data elements, and blue blocks are compare-and-swap operations}
	\label{fig:sort_odd_even}
\end{figure}  

The odd-even sort have a worst-case step complexity of $\mathcal{O}(n)$ as the maximum number of array position an element can move is $n$. The worst case work complexity is $\mathcal{O}(n^2)$ as $\mathcal{O}(n)$ operations is carried out in $\mathcal{O}(n)$ steps. The odd-even sort is step efficient when compared to serial versions, but other parallel sorting algorithms are more efficient.   