MATLAB is a scripting language which is very widely used among engineers, data scientists, researchers and more.
It is often used for algorithm development, data analysis, data visualization and mathematical modeling.
MATLAB also supports GPU computing, where functions can be executed on the GPU, and more specifically it provides mechanisms of how to interface to a CUDA GPU.

Using MATLAB with a GPU can very speed up many algorithms, especially those who compute on large data sets such as the pixels of an image.

The MATLAB profiler can be used to identify computationally heavy sections in the script. 
Such sections indicate that GPU acceleration might be useful.
The easiest way to utilize the GPU via MATLAB is by converting an input array to a GPU array with the \textit{gpuArray(input)} function call in MATLAB.
It also possible to invoke a CUDA kernel directly from MATLAB for example by using the function shown in \autoref{lst:matlab}.
\begin{lstlisting}[language=matlab,caption={Invoking CUDA kernel in MATLAB},label=lst:matlab]
	kernel = parallel.gpu.CUDAKernel( 'someCUDAKernel.ptx', 'someCUDAKernel.cu' );
\end{lstlisting}
The \textit{.ptx} is needed for MATLAB when running the kernel and can this file can be generated when compiling the \textit{.cu} file.
As can be seen CUDA can be utilized in multiple ways in MATLAB with relatively little effort.
For more information and full documentation on how to use a CUDA GPU in MATLAB see \url{www.mathworks.com/gpu}.