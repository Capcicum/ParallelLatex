Modern software programmers have been using standard processing unit architectures, such as Central Processing Units (CPUs) for decades, where highly developed compilers help utilizing the underlying hardware.
However, Graphical Processing Unit (GPU) programmers must have considerable knowledge of the underlying hardware architecture to fully achieve optimized and efficient performance results.
Achieving such knowledge can be difficult as GPU hardware architecture is complex and comes in many variations.
This is a result of the major development of which GPUs has been undergoing, from the first GPUs emerging back in the late 1990s to the newest upcoming architectures not yet released.

The following sections will provide a understanding of how GPU hardware architecture is structured and how it works.
In \cref{sec-hw-early-evolution}, a description of the early evolution of GPUs is presented, leading to the birth of the General Purpose Graphical Processing Unit (GPGPUs).
Next, \cref{sec-hw-cpu-gpu-architecture} will present the correlation between the CPU and the GPU, in addition with the PC architecture which allows communication between the two.
Hereafter, the GPU hardware architecture is described in details in \cref{sec-hw-gpu-arhchitecture}, including its memory model.
As the architecture differs depending on the manufacturer and version of GPU, the hardware architecture presented is illustrated as a conceptual GPGPU architecture.
Lastly, in \cref{sec-hw-variations} alternative GPU hardware architectures is presented, in addition with a description of the future architectures not yet launched.

 