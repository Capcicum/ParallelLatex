PyCUDA enables execution of Python code on a CUDA device from a Python-script.
PyCUDA wraps all CUDA functionality and enables execution of kernels and copying data between the host and the device.
A major advantage of using PyCUDA in Python scripts is that execution times are potentially sped up.

Typical Python structures and libraries such as the popular \textit{numpy} library can be used in conjunction with PyCUDA, which is a huge benefit for researchers and developers working in this high level scripting language but would still like to be able to boost their algorithm performance.

Nevertheless, using PyCUDA has some CUDA knowledge preliminaries.
Kernels are written as C++ code, which are parsed to the device.
This means that the user of PyCUDA needs to know how to write such kernels.
There are a lot of features and details which will not be explained here.
This section is merely here such that the reader is aware of that CUDA utilization exists in Python.
For more information of PyCUDA and how it can be used please see \url{https://documen.tician.de/pycuda/}.