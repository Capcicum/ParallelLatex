The Streaming Processors(SPs) goes under various different names including \textit{Scalar-Processors}, \textit{Execution Units} or the more famous \textit{CUDA Cores} by Nvidia.
SPs are the basic computing elements of a GPU, they perform fundamental operations, such as integer, floating-point, comparison and type conversions \cite{Li2016}.
SPs are comparable to the cores located in CPUs, containing similarities such as simultaneous multi-threading (SMT).
However, in order to increase the number of cores as seen for the GPU architecture, these SP cores need to be much simpler.
SPs are stripped from complex features normally resided in CPU cores such as:

\begin{itemize}
	\item \textbf{No Out-of-order execution,} which otherwise allows for parallel execution for the internal computational units within the core. 
	\item \textbf{No Cache memory,} instead the utilization of cache memory is moved to the SM. 
\end{itemize}

Furthermore, SPs does not contain its own computation registers, or instruction units.
Instead they depend on the Registration File and Warp Scheduler located in the SM.
Therefore SPs has to be considered as a part of a multi-processor to be compatible with the usual understanding of a computing core \cite{Maitre2013}.

By looking at the internal architecture of the SP, seen on \cref{fig:hw-sm}, it is seen that the SP contains five components:
\\\\
\textbf{Dispatch Port -} The Dispatch Port is used to issue the execution of the given SP.
\\\\
\textbf{Operand Collector -} The Operand Collector is used to load operands from the Register File and then buffer them for computation in the SP.
\\\\	
\textbf{FP Unit -} The Floating Point Unit is, by its name, used to perform floating point data instructions.
It is one of the two dedicated data paths within the SP.
Support of double-precision (64-bit) floating point units is handled different depending on the architecture version.
Some architectures splits the double-precision instruction into multiple single-precision (32-bit) instructions, where other contain separate double-precision floating point units \cite{Johansson2010}.
\\\\	
\textbf{INT Unit -} The Integer Unit, also known as Arithmetic Logic Unit (ALU), is the second data path within the SP. 
As its name indicates, the INT Unit performs integer data operations.
As the SP does not support Out-of-order execution, the two data paths cannot be active simultaneously.
\\\\	
\textbf{Result Queue -} The Result Queue can be seen as the opposing component to the Operand Collector.
It buffers output data from the FP Unit and the INT Unit, until it is written to the Register File.
However, not all operations needs to be written back to the Register File.
If the output of a operation is used as a intermediate result, which is to be consumed by another operation, it is possible to forward it directly to the other operation without saving it to the Register File.
This concept is also known as a forwarding network, used in CPUs \cite{Kanter2009}.